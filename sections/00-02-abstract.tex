\documentclass[../main.tex]{subfiles}

\begin{document}

\begin{center}
  \LARGE{\textbf{Tóm tắt}}
  \\[1cm]
\end{center}

Các mô hình nhận diện tên thực thể Y Sinh thường yêu cầu trích xuất ra các đặc trưng một cách thủ công cho từng loại thực thể như gene, tên bệnh, chất hóa học, ... để đạt kết quả tốt nhất. Mặc dù đã có một vài nghiên cứu thử nghiệm trên mạng nơ-ron để giải bài toán giúp giảm thiểu việc thủ công, kết quả của các mô hình vẫn thường giới hạn cho từng loại thực thể. Khóa luận này đưa ra mô hình học sâu đa tác vụ nhằm giải quyết bài toán này. Thử nghiệm trên 5 tập dữ liệu cho thấy kết quả có sự cải thiện so với chỉ sử dụng 1 tập dữ liệu. Có được điều này là nhờ việc học được thông tin lẫn nhau giữa các tập dữ liệu - các thông tin về kí tự và ở mức tư.

Từ khóa: Nhận diện tên thực thể, học đa tác vụ, đặc trưng ngữ pháp, BiLSTM, CRF 


\end{document}
